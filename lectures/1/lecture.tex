\documentclass[]{article}
\usepackage{lmodern}
\usepackage{amssymb,amsmath}
\usepackage{ifxetex,ifluatex}
\usepackage{fixltx2e} % provides \textsubscript
\ifnum 0\ifxetex 1\fi\ifluatex 1\fi=0 % if pdftex
  \usepackage[T1]{fontenc}
  \usepackage[utf8]{inputenc}
\else % if luatex or xelatex
  \ifxetex
    \usepackage{mathspec}
    \usepackage{xltxtra,xunicode}
  \else
    \usepackage{fontspec}
  \fi
  \defaultfontfeatures{Mapping=tex-text,Scale=MatchLowercase}
  \newcommand{\euro}{€}
\fi
% use upquote if available, for straight quotes in verbatim environments
\IfFileExists{upquote.sty}{\usepackage{upquote}}{}
% use microtype if available
\IfFileExists{microtype.sty}{\usepackage{microtype}}{}
\usepackage{graphicx}
\makeatletter
\def\maxwidth{\ifdim\Gin@nat@width>\linewidth\linewidth\else\Gin@nat@width\fi}
\def\maxheight{\ifdim\Gin@nat@height>\textheight\textheight\else\Gin@nat@height\fi}
\makeatother
% Scale images if necessary, so that they will not overflow the page
% margins by default, and it is still possible to overwrite the defaults
% using explicit options in \includegraphics[width, height, ...]{}
\setkeys{Gin}{width=\maxwidth,height=\maxheight,keepaspectratio}
\ifxetex
  \usepackage[setpagesize=false, % page size defined by xetex
              unicode=false, % unicode breaks when used with xetex
              xetex]{hyperref}
\else
  \usepackage[unicode=true]{hyperref}
\fi
\hypersetup{breaklinks=true,
            bookmarks=true,
            pdfauthor={},
            pdftitle={},
            colorlinks=true,
            citecolor=blue,
            urlcolor=blue,
            linkcolor=magenta,
            pdfborder={0 0 0}}
\urlstyle{same}  % don't use monospace font for urls
\setlength{\parindent}{0pt}
\setlength{\parskip}{6pt plus 2pt minus 1pt}
\setlength{\emergencystretch}{3em}  % prevent overfull lines
\setcounter{secnumdepth}{0}


\begin{document}

\section{Introduction}\label{introduction}

\subsection{Syllabus}\label{syllabus}

\subsubsection{Highlights}\label{highlights}

\begin{itemize}
\itemsep1pt\parskip0pt\parsep0pt
\item
  \textbf{Professor}: Max Lieblich, lieblich@uw.edu
\item
  \textbf{Course website}: \href{../Math126/index.php}{enjoy}. (Just
  search for ``Max Lieblich math 126'' on the internet to find it.)
\item
  \textbf{Homework}: \href{http://www.webassign.net}{webassign}, due
  every Tuesday and Thursday
\item
  \textbf{Midterms}: two.
\item
  \textbf{Final}: end of quarter, three hours of fun.
\item
  \textbf{Grading}: roughly 33\% for each midterm and the final. I
  reserve the right to tweak this at any time.
\item
  What can we expect from one another?
\end{itemize}

\subsection{Math Study Center}\label{math-study-center}

\begin{itemize}
\itemsep1pt\parskip0pt\parsep0pt
\item
  Open to anyone, with questions or without, confused or clear, loving
  math or not.
\item
  Communications B-014
\item
  Hours:

  \begin{itemize}
  \itemsep1pt\parskip0pt\parsep0pt
  \item
    M-Th: 9:30AM to 9:30PM
  \item
    Fri: 9:30AM to 1:30PM
  \item
    Sun: 2PM to 6PM
  \end{itemize}
\item
  You will need to make your own private Math Study Center on Saturday.
\end{itemize}

\subsection{Questions?}\label{questions}

\begin{itemize}
\itemsep1pt\parskip0pt\parsep0pt
\item
  Am I ready for this course?
\item
  What will the median grade be?
\item
  How will I ever stop loving calculus?
\end{itemize}

\subsection{You have seen}\label{you-have-seen}

\begin{itemize}
\itemsep1pt\parskip0pt\parsep0pt
\item
  Derivatives\ldots{}
\item
  Integrals\ldots{}
\item
  Differential equations\ldots{}
\item
  In one variable only
\item
  (with a smidgen of parametric motion).
\end{itemize}

\subsubsection{We have really only equipped you to understand life on a
string.}\label{we-have-really-only-equipped-you-to-understand-life-on-a-string.}

\subsection{That sucks}\label{that-sucks}

\subsubsection{How can we understand a situation closer to
reality?}\label{how-can-we-understand-a-situation-closer-to-reality}

\paragraph{How can we}\label{how-can-we}

\begin{itemize}
\itemsep1pt\parskip0pt\parsep0pt
\item
  model three-dimensional space?
\item
  describe shapes in that space?
\item
  describe physical properties of objects in space (center of mass,
  density, etc.)?
\end{itemize}

\section{Questions we might ask:}\label{questions-we-might-ask}

\subsection{Question}\label{question}

How does it feel to fly along this trefoil path?

\subsection{Question}\label{question-1}

How do we find lines perpendicular to a surface (even a weird one)?

\subsection{Question}\label{question-2}

What makes this shape\ldots{}

\subsection{Question}\label{question-3}

\ldots{}different from this one?

\subsection{Properties we might
examine}\label{properties-we-might-examine}

\subsubsection{We could try to characterize shapes and objects using
things
like}\label{we-could-try-to-characterize-shapes-and-objects-using-things-like}

\begin{itemize}
\itemsep1pt\parskip0pt\parsep0pt
\item
  Curvature (what is this?)
\item
  Surface area (I think I know what this is)
\item
  Volume (OK, whatever)
\item
  What other mathematical properties might distinguish objects?
\end{itemize}

\section{Space}\label{space}

\subsection{What is reality?}\label{what-is-reality}

\subsubsection{Three dimensional space}\label{three-dimensional-space}

\begin{itemize}
\itemsep1pt\parskip0pt\parsep0pt
\item
  What is it?
\item
  Here's a picture:
\item
  No, really, that's a picture. Is it missing something?
\end{itemize}

\subsection{What is reality?}\label{what-is-reality-1}

\subsubsection{How can we describe this space so that we can calculate
things? Get a handle on it? Use it for
something?}\label{how-can-we-describe-this-space-so-that-we-can-calculate-things-get-a-handle-on-it-use-it-for-something}

\begin{itemize}
\itemsep1pt\parskip0pt\parsep0pt
\item
  Predict future positions or motions
\item
  Quantify mass, volume, stress
\item
  Tell the supplier how much cheese we need for the giant wheel
\end{itemize}

\subsection{René Descartes}\label{renuxe9-descartes}

\begin{itemize}
\itemsep1pt\parskip0pt\parsep0pt
\item
  Je pense, donc je suis.
\item
  ``I think, therefore I am.''
\end{itemize}

\subsubsection{Descartes thought of something brilliant, something that
shook the
world.}\label{descartes-thought-of-something-brilliant-something-that-shook-the-world.}

\begin{itemize}
\itemsep1pt\parskip0pt\parsep0pt
\item
  Does anyone know what I am talking about?
\item
  (The pineal gland??)
\end{itemize}

\subsection{René Descartes}\label{renuxe9-descartes-1}

\subsubsection{Descartes discovered
\emph{coordinates}}\label{descartes-discovered-coordinates}

\begin{itemize}
\itemsep1pt\parskip0pt\parsep0pt
\item
  The 3D space of human experience is the set of ordered triples of
  numbers:
\item
  \[\mathbf{R^3}=\{(x,y,z) | x,y,z\in\mathbf{R}\}\]
\item
  Here's a picture you probably recognize.
\end{itemize}

\includegraphics{cartesian.png}

\subsection{Numbers breed numbers}\label{numbers-breed-numbers}

\subsubsection{We can now calculate
distance!}\label{we-can-now-calculate-distance}

Distance between two points $(a,b,c)$ and $(a',b',c')$ is

\[\sqrt{(a'-a)^2+(b'-b)^2+(c'-c)^2}.\]

This generalizes the Pythagorean theorem. The book has a good
explanation of why it's true. See if you can figure it out (using the
Pythagorean theorem) before you read it! If you have already read it,
try before reading it again. (You read each section of the book several
times, right?)

\subsection{Numbers breed equations}\label{numbers-breed-equations}

\subsubsection{We can now describe
shapes!}\label{we-can-now-describe-shapes}

\begin{itemize}
\itemsep1pt\parskip0pt\parsep0pt
\item
  What is the set of points at distance 1 from $(0,0,0)$?
\item
  What shape is the set of points $(x,y,z)$ such that $x+y=z$?
\item
  \ldots{}such that $x^2+y^2=z^2$?
\item
  \ldots{}such that $x^2+y^2=z$? (How does it differ from the previous
  one?)
\item
  \ldots{}such that $y=x^2$?
\item
  \ldots{}such that $z=4$?
\end{itemize}

\end{document}
